
% This LaTeX was auto-generated from MATLAB code.
% To make changes, update the MATLAB code and republish this document.

\documentclass{article}
\usepackage{graphicx}
\usepackage{color}

\sloppy
\definecolor{lightgray}{gray}{0.5}
\setlength{\parindent}{0pt}

\begin{document}

    
    
\subsection*{Contents}

\begin{itemize}
\setlength{\itemsep}{-1ex}
   \item Physical constants
   \item Pitch closed loop syntesis
   \item Elevation closed loop analysis
\end{itemize}
\begin{verbatim}
% Initialization for the helicopter assignment in TTK4135.
% Run this file before you execute QuaRC -> Build.

% Updated spring 2017, Andreas L. Fl�ten

%clear all;
close all;
clear all
clc;
\end{verbatim}


\subsection*{Physical constants}

\begin{verbatim}
m_h = 0.4; % Total mass of the motors.
m_g = 0.03; % Effective mass of the helicopter.
l_a = 0.65; % Distance from elevation axis to helicopter body
l_h = 0.17; % Distance from pitch axis to motor

% Moments of inertia
J_e = 2 * m_h * l_a *l_a;         % Moment of interia for elevation
J_p = 2 * ( m_h/2 * l_h * l_h);   % Moment of interia for pitch
J_t = 2 * m_h * l_a *l_a;         % Moment of interia for travel

% Identified voltage sum and difference
V_s_eq = 6.35; % Identified equilibrium voltage sum.
V_d_eq = 0.3; % Identified equilibrium voltage difference.

% Model parameters
K_p = m_g*9.81; % Force to lift the helicopter from the ground.
K_f = K_p/V_s_eq; % Force motor constant.
K_1 = l_h*K_f/J_p;
K_2 = K_p*l_a/J_t;
K_3 = K_f*l_a/J_e;
K_4 = K_p*l_a/J_e;
\end{verbatim}


\subsection*{Pitch closed loop syntesis}

\begin{par}
Controller parameters
\end{par} \vspace{1em}
\begin{verbatim}
w_p = 1.8; % Pitch controller bandwidth.
d_p = 1.0; % Pitch controller rel. damping.
K_pp = w_p^2/K_1;
K_pd = 2*d_p*sqrt(K_pp/K_1);
Vd_ff = V_d_eq;

% Closed loop transfer functions
Vd_max = 10 - V_s_eq; % Maximum voltage difference
deg2rad = @(x) x*pi/180;
Rp_max = deg2rad(15); % Maximum reference step
s = tf('s');
G_p = K_1/(s^2);
C_p = K_pp + K_pd*s/(1+0.1*w_p*s);
L_p = G_p*C_p;
S_p = (1 + L_p)^(-1);

plot_pitch_response = 0;
if plot_pitch_response
    figure()
    step(S_p*Rp_max); hold on;
    step(C_p*S_p*Rp_max/Vd_max);
    legend('norm error', 'norm input')
    title('Pitch closed loop response')
end
\end{verbatim}


\subsection*{Elevation closed loop analysis}

\begin{par}
Controller parameters
\end{par} \vspace{1em}
\begin{verbatim}
w_e = 0.5; % Elevation controller bandwidth.
d_e = 1.0; % Elevation controller rel. damping.
K_ep = w_e^2/K_3;
K_ed = 2*d_e*sqrt(K_ep/K_3);
K_ei = K_ep*0.1;
Vs_ff = V_s_eq;

% Closed loop transfer functions
Vs_max = 10 - V_s_eq; % Maximum voltage sum
Re_max = deg2rad(10); % Maximum elevation step
G_e = K_3/(s^2);
C_e = K_ep + K_ed*s/(1+0.1*w_e*s) + K_ei/s;
L_e = G_e*C_e;
S_e = (1 + L_e)^(-1);

plot_elev_response = 0;
if plot_elev_response
    figure()
    step(S_e*Re_max);
    hold on;
    step(C_e*S_e*Re_max/Vs_max);
    legend('norm error', 'norm input')
    title('Elevation closed loop response')
end

%Code task 4.1
I = eye(6);
A_c = [0 1 0 0 0 0; 0 0 -K_2 0 0 0; 0 0 0 1 0 0; 0 0 -K_1*K_pp -K_1*K_pd 0 0; 0 0 0 0 0 1; 0 0 0 0 -K_3*K_ep -K_3*K_ed];
B_c = [0 0; 0 0; 0 0; K_1*K_pp 0; 0 0; 0 K_3*K_ep];
Theta_t = 0.25;
A = I + Theta_t*A_c;
B = Theta_t*B_c;

%Constraint
%P_k = 30*pi/180;

%Optimal trajectory
X_0 = [pi; 0; 0; 0];
X_f = [0; 0; 0; 0;];
\end{verbatim}



\end{document}
    
